\chapter{Конструкторская часть}

В данном разделе представлены этапы проектирования программного обеспечения.

\section{Состав программного обеспечения}

Программное обеспечение состоит из клиент-серверного приложения. Структура разрабатываемого проекта:

\begin{itemize}
	\item сервер -- программа, обрабатывающая запросы клиентов и являющаяяся посредником для передачи информации от одного клиента всем остальным;
	\item клиент -- программа, которая является инициатором соединения и способная генерировать события.
\end{itemize}

В данной работе сервер -- это программа, принимающая от пользователей сообщения (или команды), отправленные через клиент, и отправляющая другим пользователям принятые сообщения (шаблон издатель-подписчик), а клиент -- программа, принимающая сообщения (или команды) от пользователя через устройство ввода и отправляющая полученные данные на сервер. Также клиент принимает сообщения от сервера. На основе этих сообщений строится логика взаимодействия пользователя с программой. Сервер в данном программном обеспечении выступает только как брокер сообщений.

\section{Сценарий использования}

На рисунках \ref{img:cp_scenario_client} и \ref{img:cp_scenario_server}  представлены сценарии взаимодействия пользователя с клиентским приложением и сервера с полученным сообщением соответственно.

\img{80mm}{cp_scenario_client}{Сценарий взаимодействия пользователя с клиентским приложением}

\img{80mm}{cp_scenario_server}{Сценарий взаимодействия сервера с полученным сообщением}

\section{Проектирование зон ответственности компонентов}

В данном программном обеспечении сервер играет роль только брокера сообщений. Вся логика их обработки сосредоточена на клиенте. В зависимости от полученного сообщения от сервера клиент выполняет различные функции:

\begin{itemize}
	\item если поступила команда (сообщение, которое начинается с символа \texttt{/}) -- выполнить действие на основе команды (создание комнаты, смена комнаты, выход из комнаты, загрузка или отправка файла, получение информации о файлах);
	\item если поступил обычный текст -- отобразить данный текст в окне чата.
\end{itemize}

Информация о существующих чатах, пользователях чата и файлах чата находится на сервере. Информация о чатах пользователя хранится только на клиенте в момент работы программы.

Сервер не хранит историю сообщений, так как реализован по модели издатель-подписчик. Как только сообщение попадает на сервер, оно тут же рассылается всем подписанным клиентам и никак не запоминается на сервере.

В случае, если пользователь находится в нескольких чатах одновременно, клиент на своей стороне хранит те сообщения, которые были получены в неактивных чатах (все чаты кроме открытого на данный момент считаются неактивными) и показывает их в тот момент, когда пользователь меняет активный чат на один из неактивных. Сообщения, которые были показаны до смены активного чата, показаны не будут.

\section*{Вывод}

В данном разделе были представлены сценарии взаимодействия пользователя с клиентом и сервера с полученными сообщениями. Были спроектированы зоны ответственности клиентской и серверной части приложения.