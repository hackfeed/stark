\chapter{Аналитическая часть}

В данном разделе представено описание существующих протоколов обмена сообщениями, анализ существующих решений и выбор протокола для реализации в работе.

\section{Постановка задачи}

В соответствии с заданием необходимо разработать программный комплекс, реализующий обмен текстовыми сообщениями между пользователями в режиме реального времени. Одновременно к приложению может быть подключено несколько пользователей. Предусмотреть возможность создавать комнаты для общения, подключаться к уже существующим комнатам по приглашению администратора (или администраторов) комнаты. При подключении к комнате не показывать новому пользователю историю сообщений. Предоставить пользователям возможность отправлять файлы. Реализовать консольный интерфейс для приложения. Для решения этой задачи необходимо изучить предментную область и проанализировать существующие решения.

\section{Протоколы обмена мгновенными сообщениями}

Система мгновенного обмена сообщениями \cite{im} (англ. Instant messaging, IM) — система для обмена сообщениями в реальном времени через Интернет. В таких системах могут передаваться текстовые сообщения, звуковые сигналы, изображения, видео, а также производиться такие действия, как совместное рисование или игры. Многие из таких программ-клиентов могут применяться для организации групповых текстовых чатов или видеоконференций.

Системы мгновенного обмена сообщениями используют соответствующие протоколы или их модификации.

В данном подразделе буду рассмотрены открытые протоколы мгновенного обмена сообщениями.

\subsection{Distributed Data Protocol}

Distributed Data Protocol или DDP \cite{ddp} (протокол распределенных данных) -- протокол клиент-серверного взаимодействия, созданный для использования инфраструктурой JavaScript веб-платформы Meteor и использующий в качестве обмена сообщениями шаблон издатель-подписчик \cite{pubsub}.

Стандартным способом передачи данных через DDP является передача EJSON через веб-сокеты. Вторым вариантом является использование Long Poll.

В случае, если браузер не поддерживает веб-сокеты, передача данных будет осуществляться с использованием Long Poll.

\subsection{IRC}

IRC \cite{irc} (англ. Internet Relay Chat) -- протокол прикладного уровня для обмена сообщениями в режиме реального времени. Разработан в основном для группового общения, также позволяет общаться через личные сообщения и обмениваться данными, в том числе файлами.

IRC использует транспортный протокол TCP и криптографический TLS (опционально).

IRC предоставляет возможность как группового, так и приватного общения. Для группового общения существует несколько возможностей. Пользователь может отправить сообщение списку пользователей, при этом серверу отправляется список, сервер выделяет из него отдельных пользователей и отправляет копию сообщения каждому из них.

Более эффективным является использование каналов. В этом случае сообщение отправляется непосредственно серверу, а сервер отправляет его всем пользователям в канале.

Как при групповом, так и при приватном общении сообщения отправляются клиентам по кратчайшему пути и видимы только отправителю, получателю и входящим в кратчайший путь серверам.

Кроме того, возможна отправка широковещательного сообщения. Сообщения клиентов, касающиеся изменения состояния сети (например, режима канала или статуса пользователя), должны отправляться всем серверам, входящим в сеть. Все сообщения, исходящие от сервера, также должны быть отправлены всем остальным серверам.

\subsection{Matrix}

Matrix \cite{matrix} -- открытый протокол мгновенного обмена сообщениями и файлами с поддержкой голосовой и видеосвязи. Это децентрализованный клиент-серверный протокол с передачей сообщений между серверами.

Протокол Matrix позиционирован создателями как замена для более ранних протоколов, он призван объединить мгновенные сообщения с голосовым и видео-общением, что не удалось сделать в рамках SIP, XMPP и RCS.

Ключевые особенности протокола Matrix -- объединение в одном месте всех каналов непосредственного общения и децентрализация.

Концепция Matrix основана на принципах построения электронной почты. Внутренняя организация протокола похожа на IRC -- доверенные серверы обмениваются сообщениями чатов друг с другом. При этом Matrix отличается от того же IRC низким порогом вхождения, для общения через Matrix не нужно быть опытным пользователем, идентификация проста и осуществляется по номеру телефона, адресу электронной почты, аккаунтам Facebook или Google или другим способом, привычным пользователю.

\subsection{XMPP}

XMPP \cite{xmpp} (англ. eXtensible Messaging and Presence Protocol <<расширяемый протокол обмена сообщениями и информацией о присутствии>>) -- открытый, основанный на XML, свободный для использования протокол для мгновенного обмена сообщениями и информацией о присутствии в режиме, близком к режиму реального времени. Изначально спроектированный легко расширяемым, протокол, помимо передачи текстовых сообщений, поддерживает передачу голоса, видео и файлов по сети.

Расширяемость протокола предназначена для добавления в единую коммуникационную сеть мессенджеров, социальных сетей, сайтов, использующих разные, несовместимые стандарты. Предполагалось, что крупные компании будут открывать межсерверное общение с другими IM и описывать свои методы шифрования, передачи мультимедиа и других данных через публикацию расширений XMPP. Расширения будут приниматься или отклоняться глобальным сообществом путём наибольшего распространения, но при этом всегда будет доступна базовая функциональность для передачи сообщений для пользователей разных мессенджеров. В реальности данная идея не получила должного распространения, и большинство крупных компаний не стало открывать возможность коммуникации для своих пользователей с другими сервисами.

В отличие от коммерческих систем мгновенного обмена сообщениями, таких как AIM, ICQ, WLM и Yahoo, XMPP является федеративной, расширяемой и открытой системой. Любой желающий может запустить свой сервер мгновенного обмена сообщениями, регистрировать на нём пользователей и взаимодействовать с другими серверами XMPP.

\section{Выбор протокола для решения задачи}

В качестве реализуемого в работе протокола, будет использован модифицированный протокол IRC с использованием шаблона издатель-подписчик. Данный выбор обоснован наиболее подходящим поведением протокола для реализации групповых чатов.

\section{Модель клиент-сервер}

В модели клиент-сервер роли определены: сервер предоставляет ре­сурсы и службы одному или нескольким клиентам, которые обращаются к серверу за обслуживанием. В качестве примеров серверов можно привести веб­-серверы, почтовые серверы и файловые серверы . Каждый из этих серверов предоставляет ресурсы для клиентских устройств, таких как настольные ком­пьютеры, ноутбуки, планшеты и смартфоны . Большинство серверов могут устанавливать отношение <<один ко многим>> с клиентами, что означает, что один сервер может предоставлять ресурсы нескольким клиентам одновремен­но. Когда клиент запрашивает соединение с сервером, сервер может либо принять, либо отклонить это соединение. Если соединение принято, сервер устанавливает и поддерживает соединение с клиентом по определенному про­токолу. Например, почтовый клиент может запросить SMTP-соединение с почтовым сервером для отправки сообщения. Затем приложение SMTP на почтовом сервере запросит проверку подлинности у клиента, например адрес электронной почты и пароль. Если эти учетные данные совпадают с учетной записью на почтовом сервере, сервер отправит электронное письмо целевому получателю. Часто клиенты и серверы взаимодействуют через компьютерную сеть на разных аппаратных средствах, но и клиент и сервер могут находиться в одной и той же системе. Хост сервера запускает одну или несколько серверных программ, которые совместно используют свои ресурсы с клиентами. Клиент не предоставляет общий доступ ни к одному из своих ресурсов, но запраши­вает данные или службу у сервера. Поэтому клиенты инициируют сеансы связи с серверами, которые ожидают входящих запросов. Клиенту не знает о том, как работает сервер при выполнении запроса и доставке ответа. Клиент должен только понимать ответ, основанный на хорошо известном прикладном протоколе, т.е. содержание и форматирование данных для запрашиваемой службы. Клиенты и серверы обмениваются сообщениями в шаблоне обмена сообщениями запрос-ответ. Клиент отправляет запрос, а сервер возвращает ответ.

\section{Анализ существующих решений}

В качестве существующих решений для анализа выбраны сервисы \texttt{cli-chat} \cite{clichat}, \texttt{go-cli-chat} \cite{goclichat}, \texttt{crio} \cite{crio} и \texttt{chattt} \cite{chattt}.

В таблице \ref{tab:solutions} представлен сравнительный анализ существующих решений.

\begin{table}[H]
	\centering
	\caption{Анализ существующих решений}
	\label{tab:solutions}
	\resizebox{\textwidth}{!}{%
		\begin{tabular}{|c|c|c|c|c|}
			\hline
			\textbf{Сервис} & \textbf{\begin{tabular}[c]{@{}c@{}}Максимальное\\ количество комнат\end{tabular}} & \textbf{\begin{tabular}[c]{@{}c@{}}Обработка\\ непрочитанных \\ сообщений\end{tabular}} & \textbf{\begin{tabular}[c]{@{}c@{}}Максимальное\\ количество \\ пользователей\end{tabular}} & \textbf{\begin{tabular}[c]{@{}c@{}}Передача \\ файлов\end{tabular}} \\ \hline
			cli-chat        & Не ограничено                                                                     & Нет                                                                                     & Не ограничено                                                                               & Нет                                                                 \\ \hline
			go-cli-chat     & 1                                                                                 & Нет                                                                                     & Не ограничено                                                                               & Нет                                                                 \\ \hline
			crio            & Не ограничено                                                                     & Нет                                                                                     & Не ограничено                                                                               & Нет                                                                 \\ \hline
			chatt           & 1                                                                                 & Нет                                                                                     & Не ограничено                                                                               & Нет                                                                 \\ \hline
		\end{tabular}%
	}
\end{table}

\section*{Вывод}

В данном разделе были рассмотрены и проанализированы протоколы мгновенного обмена сообщения, также выбран протокол, на основе которого будет реализовано программное обеспечение. В качестве основы для протокола был выбран протокол IRC в виду наличия поддержки реализации группового обмена сообщениями.