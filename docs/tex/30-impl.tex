\chapter{Технологическая часть}

В данном разделе представлены средства разработки программного обеспечения, детали реализации и пользовательский интерфейс.

\section{Средства реализации}

Для разработки клиентской части приложения был выбран язык \texttt{Go} \cite{golang}. Выбор обусловлен тем, что данный язык является компилируемым, следовательно в процессе разработке возникнет меньше промежуточных ошибок. Также Go предоставляет нативную поддержку многопоточности, что позволяет повысить производительность приложения. В качестве библиотеки для реализации пользовательского интерфейса была выбрана библиотека \texttt{gocui} \cite{gocui}. Выбор библиотеки обусловлен опытом работы с данным инструментом.

Для реализации серверной части приложения была выбрана СУБД \texttt{Redis} \cite{redis}. Данная СУБД может выступить брокером сообщений нативно, без дополнительных надстроек, так как изначально подерживает шаблон издатель-подписчик \cite{redispubsub} и имеет интерфейс, реализующий его работу. Кроме того, ее можно использовать для хранения информации о чатах (пользователи, файлы).

\section{Детали реализации}

В листингах \ref{lst:db} -- \ref{lst:handlers} представлены листинги реализации интерфейса взаимодействия клиента с брокером сообщений (сервером), обработка полученных сообщений от сервера клиентом и функции обработки команд и обновления состояния на стороне клиента.

\begin{lstinputlisting}[
	caption={Реализация интерфейса взаимодействия клиента с брокером сообщений},
	label={lst:db},
	style={go},
	linerange={27-83},
	]{/Users/sekononenko/Study/stark/internal/db/cache/redis.go}
\end{lstinputlisting}

\begin{lstinputlisting}[
	caption={Обработка полученных сообщений от сервера клиентом},
	label={lst:client},
	style={go},
	linerange={83-287},
	]{/Users/sekononenko/Study/stark/internal/client/client.go}
\end{lstinputlisting}

\begin{lstinputlisting}[
	caption={Обработка команд и обновление состояния на стороне клиента},
	label={lst:handlers},
	style={go},
	linerange={17-201},
	]{/Users/sekononenko/Study/stark/internal/client/handler.go}
\end{lstinputlisting}

\section{Пользовательский интерфейс}

На рисунках \ref{img:cp_demo0} -- \ref{img:cp_demo2} представлены примеры работы программного обеспечения.

\img{90mm}{cp_demo0}{Отправка сообщения в чат и подключение нового пользователя}

\img{90mm}{cp_demo1}{Подключение нового пользователя}

\img{90mm}{cp_demo2}{Смена комнаты и загрузка файла}

\section*{Вывод}

В данном разделе были представлены средства реализации программного обеспечения, листинги ключевых компонентов системы а также представлен пользовательский интерфейс приложения.
