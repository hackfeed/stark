\chapter*{Введение}
\addcontentsline{toc}{chapter}{Введение}

Системы обмена сообщениями устойчиво закрепились в жизни человека. Они используются не только для общения с друзьями или близкими людьми, но и для общения по работе или учебе. Особенностью нынешних систем обмена текстовыми сообщениями является возможность создавать группы, в которых могут общаться сразу несколько человек. Данная особенность избавляет пользователя от необходимости дублировать информацию нескольким людям. Кроме того в группах можно делиться файлами с участниками групп, что также упрощает взаимодействие, избавляя человека от необходимости создания электронного письма с вложением или загрузки файла на удаленный сервер.

Цель работы -- разработать программный комплекс, реализующий обмен текстовыми сообщениями между пользователями в режиме реального времени.

Чтобы достигнуть поставленной цели, требуется решить следующие задачи:
\begin{itemize}
	\item провести анализ существующих решений;
	\item проанализировать протоколы обмена сообщениями;
	\item реализовать в программном комплексе протокол обмена сообщениями;
	\item реализовать программное обеспечение для обмена сообщениями в режиме реального времени.
\end{itemize}